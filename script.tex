\documentclass[a4paper]{article}

%--------------------------------------- Packages
\usepackage{mathtools}
\usepackage{amssymb}
\usepackage{amsthm}
\usepackage{enumitem}
\usepackage{bbold}

%--------------------------------------- Definitions
\newcommand{\prob}{\mathbf{P}}
\newcommand{\expe}{\mathbf{E}}
\newcommand{\real}{\mathbb{R}}
\newcommand{\prty}{\mathcal{P}}
\newcommand{\nset}{[n]}
\newcommand{\eps}{\varepsilon}

%--------------------------------------- Environments
\theoremstyle{plain}
\newtheorem{theorem}{Theorem}
\newtheorem{lemma}{Lemma}
\theoremstyle{definition}
\newtheorem{definition}{Definition}

%--------------------------------------- Document
\begin{document}
\title{Testing of Boolean functions}
\author{Pascal Huber}
\date{\today}
\maketitle

\section{Motivation}
\label{sec:motivation}

\section{Reminder}
\label{sec:reminder}

\paragraph{Boolean functions}
\label{sec:boolean-functions}

We consider functions from the hypercube \(\Omega_n \coloneqq
\{-1,1\}^n\) into the real numbers. A function \(f: \Omega_n
\rightarrow \real\) is called \emph{Boolean function} if it takes values only in
\(\{-1,1\}\). \\\\
We recall two important types of Boolean functions:\\
\begin{enumerate}
\item Let \(\nset \coloneqq \left\{1,2, \dots, n\right\}\) and let
  \(S\subseteq \nset\). Then the \emph{parity function over S} is
  defined by 
  \[ \chi_S(x) \coloneqq x_S \coloneqq \prod_{i \in S} x_i. \]
  Note that the product over the empty set is equal to one: \(\prod_{i
    \in \emptyset} x_i = 1\). 

\item The parity functions over the sets \(S = \{i\}\) are called
  \emph{dictator functions}. They depend solely on the \(i\)-th bit,
  i.e. 
  \[\chi_i \coloneqq \chi_{\{i\}} = x_i.\] 
\end{enumerate}

\noindent\emph{(Note over the Great notational switch)?}
\\

\paragraph{Correlation of Boolean functions}
\label{sec:corr-bool-funct}

\noindent The measurable space \(\left(\Omega_n, \mathcal{P}(\Omega_n)\right)\)
is equipped with the uniform probability measure \[\prob_{x \in
  \{-1,1\}^n} = \left(\frac{1}{2} \delta_{-1} +
  \frac{1}{2}\delta_1\right)^{\otimes n}.\]
Furthermore we define the corresponding \(L^2\)-inner product (called \emph{correlation}) for two
functions \(f,g: \Omega_n \rightarrow \real\) on this space by 
\[\langle f,g\rangle \coloneqq \expe_{x \in
  \{-1,1\}^n}\left[f(x)g(x)\right].\]
Note that if \(f\) and \(g\) are Boolean functions then one always
gets \(\langle f, g\rangle \in [-1,1]\) and \(\|f\| \coloneqq
\sqrt{\langle f, f\rangle } = 1\).  

\paragraph{Results from Fourier analysis of Boolean functions}
\label{sec:results-from-fourier}

Before introducing the notions of testability of Boolean functions we
recall the most important results of the Fourier analysis of Boolean
functions. 
\begin{itemize}
\item The parity functions \(\left(\chi_S\right)_{S \subseteq \nset}\)
  form an orthonormal basis on the space of all functions from
  \(\Omega_n\) to the real numbers, i.e. 
  \[\langle \chi_S, \chi_T\rangle =  \begin{cases}
    1 & S=T \\
    0 & \text{else}
  \end{cases}\]
\item Consequently every Boolean function \(f\) has a unique
  representation of the form 
  \begin{equation}
    \label{eq:1}
    f = \sum_{S\subseteq \nset} \hat{f}(S) \chi_S.
  \end{equation}
  The coefficients \(\hat{f}(S)\) are called \emph{Fourier
    coefficients} and (\ref{eq:1}) is called the \emph{Fourier
    expansion} of \(f\).
\item \emph{Plancharel's theorem:} Let \(f,g: \Omega_n \rightarrow
  \real\). Then \(\langle f, g\rangle = \sum_{S\subseteq \nset}
  \hat{f}(S)\hat{g}(S)\).
\item \emph{Parseval's theorem:} Let \(f: \Omega_n \rightarrow
  \real\). Then \(\langle f, f\rangle = \sum_{S\subseteq \nset}
  \hat{f}(S)^2\). Especially, if \(f\) is Boolean one has the identity \(\sum_{S\subseteq \nset}
  \hat{f}(S)^2 = 1\). 
\end{itemize}


\section{Linearity of Boolean functions and basics of property testing}
\label{sec:line-bool-funct}

\subsection{Linearity of Boolean functions}
\label{sec:line-bool-funct-1}

As a motivation we will first define what it means for a Boolean
function to be linear. We introduce two possible definitions: 

\begin{definition}[Linearity of Boolean functions] \label{def:1}
  A Boolean function \(f: \{0,1\}^n \rightarrow \{0,1\}\) is called \emph{linear}
  iff one of the following statements hold: 
  \begin{enumerate}[label= (\roman*)]
  \item For all \(x,y \in \{0,1\}^n\) one has \(f(x+y) = f(x) + f(y)\).
  \item \(f\) is a parity function, i.e. there exists \(S \subseteq
    \nset\) such that for all \(x \in \{0,1\}^n\) one has \(f(x) =
    \chi_S(x)\). 
  \end{enumerate}
\end{definition}

\noindent In multiplicative notation the definition reads as follows: 
\begin{enumerate}[label= (\roman*)]
\item For all \(x,y \in \{-1,1\}^n\) one has \(f(x\circ y) = f(x)\cdot
  f(y)\), where \(x\circ y \coloneqq (x_1y_1, \dots, x_ny_n)\).
\item There exists \(S \subseteq \nset\) such that for all \(x\in
  \{-1,1\}^n\) one has \(f(x) = \prod_{i\in S} x_i\). 
\end{enumerate}

\noindent In order to make sure that Definition
\ref{def:line-bool-funct} is well defined we have to show that (i) and
(ii) are equivalent: \\
\underline{(ii) \(\Rightarrow\) (i):} is straightforward since 
\[\chi_S(x+y) = \sum_{i\in S} (x_i + y_i) = \sum_{i\in S} x_i +
\sum_{i\in S} y_i = \chi_S(x) + \chi_S(y).\]
\underline{(i) \(\Rightarrow\) (ii):} uses the representation \(x =
x_1 e_1 + \dots + x_n e_n\) where \(e_i = (0,\dots,0,1,0,\dots,0)\)
with the \(1\) at the \(i\)-th bit. Then by iterating (i) one gets
\[f(x) = f\left(\sum_{i=1}^{n} x_i e_i\right) = \sum_{i=1}^{n} x_i f(e_i) =
\sum_{i\in S} x_i, \]
where \(S \coloneqq \left\{i\in \nset : f(e_i) = 1 \right\}\). \\

\noindent Thus we see that the linear Boolean functions are exactly
the parity functions. (Note that we showed this using the additive
notation, but this is of course also true when we use the
multiplicative representation.) 

\subsection{Approximate linearity and basic notions of property testing}
\label{sec:appr-line-basic}

In the following we illustrate the basic notions of property testing
using the example of linearity testing. 

\paragraph{Approximate linearity}
\label{sec:appr-line}

The requirement to be linear on \(\{0,1\}^n\) is rather expensive to
check (one has to do all \(2^n\) queries) and hence often too
restrictive. That's why one considers a more relaxed concept of
linearity: \\

\noindent A Boolean function \(f:\{0,1\}^n \rightarrow \{0,1\}\) is called
\emph{approximate linear} iff
\begin{enumerate}[label= (\roman*')]
\item for ``most'' pairs \(x,y \in \{0,1\}^n\) one has \(f(x+y) = f(x)
  + f(y)\).
\item there exist \(S \subseteq \nset\) s.t. for ``most'' \(x\in
  \{0,1\}^n\) one has \(f(x) = \chi_S(x)\). 
\end{enumerate}
(We will define more precisely what is meant by ``most'' in a
second.)\\

\noindent As before we would like both definitions to be equivalent
and one easily sees by copying the above argument that (ii') indeed
implies (i'). The other implication on the other hand is much less
obvious. \\
A main result of this course will be that in fact (i') implies
(ii'). In order to show this we will use the results of the Fourier
analysis extensively. But before dwelling on the proof we want to put
the problem in the more general context of property testing. Therefore
we will need some definitions. 

\begin{definition}[Property of Boolean functions]
  A \emph{property of Boolean functions} is a subset \(\prty\) of the
  set of all Boolean functions. We say that a Boolean function \(f\) \emph{has property \(\prty\)} if \(f
  \in \prty\). 
\end{definition}

\begin{definition}[\(\eps\)-far and \(\eps\)-close]
  \begin{enumerate}[label=(\roman*)]
  \item Two Boolean functions \(f,g\) are \(\eps\)-close if they agree
    on a \((1-\eps)\)-fraction of \(\{0,1\}^n\), i.e. 
    \[\prob_{x\in\{0,1\}^n}\left[f(x) = g(x)\right] \geq (1-\eps).\]
    Otherwise they are \(\eps\)-far. 
  \item A Boolean function \(f\) is \(\eps\)-close to having property
    \(\prty\) if there exists some function \(g\in \prty\) such that
    \(f\) and \(g\) are \(\eps\)-close. 
  \end{enumerate}
\end{definition}

\noindent In our case the property we are interested in is to being
linear, i.e. \(\prty_{lin} \coloneqq \{\chi_S: S \subseteq
\nset\}\). \\ 
Thus (ii') can be reformulated as being \(\eps\)-close to being
linear. \\

How can (i') be understood in this context? It is favorable to
interpret it as a property test: 

\emph{Multiplicative notation of BLR test !!}

\begin{definition}[\textsf{BLR} linearity test (Blum, Luby, Rubinfeld)]
  Given blackbox access to a Boolean function \(f\) do the following
  steps: 
  \begin{enumerate}
  \item Pick \(x\) and \(y\) independently and uniformly at random
    from \(\{0,1\}^n\).
  \item Query \(f\) on \(x\), \(y\) and \(x+y\).
  \item ``Accept'' iff \(f(x) + f(y) = f(x+y)\). 
  \end{enumerate}
\end{definition}

\noindent Using this definition (i') says that the probability of
\textsf{BLR} accepting the function \(f\) is large, more precisely we
want to have \[\prob_{x,y \in \{0,1\}^n}\left[\textsf{BLR}(f) \text{
    accepts }\right] \leq (1-\eps).\]

\noindent Hence if we can prove that (i') implies (ii') then we have
shown that for the linearity property there exists a querying
algorithm making only \(3\) queries such that whenever \(f\) is
\(\eps\)-far from being linear then the algorithm accepts \(f\) with
probability of at most \(1-\eps\). \\
The existence of such a querying algorithm can also be shown for other
properties and motivates the following definition: 

\begin{definition}[Locally testable property]
  A property \(\prty\) of Boolean functions is called \emph{locally
    testable} if there exists a randomized querying algorithm
  \(\mathcal{T}\) making at most \(\mathcal{O}(1)\) queries such that: 
  \begin{enumerate}[label=(\roman*)]
  \item If \(f \in \prty\) then \(\prob[\mathcal{T}(f) \text{ accepts}] = 1\).
  \item If \(f\) is \(\eps\)-far from having property \(\prty\) then
    \(\prob[\mathcal{T}(f) \text{ accepts}] \leq 1 - \Omega(\eps)\).
  \end{enumerate}
\end{definition}

\noindent A more general definition of testability which relates the
closeness to the property with the number of queries is due to
Rubinfeld and Sudan: 

\begin{definition}[Testable property]
  A property \(\prty\) of Boolean functions is \emph{testable with
    \(q(\eps)\) queries} if there exists a randomized algorithm
  \(\mathcal{T}\) (which gets \(\eps\) as input) such that for all
  \(\eps > 0\) it makes \(q(\eps)\) queries and satisfies:
  \begin{enumerate}[label=(\roman*)]
  \item If \(f \in \prty\) then \(\prob[\mathcal{T}(f) \text{
      accepts}] \geq \frac{2}{3}\).
  \item If \(f\) is \(\eps\)-far from \(\prty\) then \(\prob[\mathcal{T}(f) \text{
      accepts}] \leq \frac{1}{3}\).
  \end{enumerate}
\end{definition}

\emph{Note about the 1/3 thing...}\\

\noindent The aim of today's course will be to show that the property
\(\prty_{lin}\) of being linear is 
\begin{itemize}
\item is locally testable (this is the implication (i') \(\Rightarrow\)
  (ii'))
\item is testable with \(\mathcal{O}\left(\frac{1}{\eps}\right)\)
  queries. 
\end{itemize}


\section{Linearity testing of Boolean functions}
\label{sec:line-test-bool}

We will prove the (local) testability of linearity in three steps: 
\begin{enumerate}
\item Express the ``acceptance probability'' of the \textsf{BLR}-test
  for an arbitrary Boolean function \(f\) in terms of its Fourier
  coefficients.
\item Prove local testability of linearity using this representation.
\item Prove testability by executing the \textsf{BLR}-test multiple
  times. 
\end{enumerate}

\noindent Since we want to use the Fourier expansion of Boolean
functions we will now switch to the multiplicative notation. 

\begin{lemma}[Acceptance probability of the \textsf{BLR}-test] \label{lem:1}
  Let \(f: \{-1,1\}^n \rightarrow \{-1,1\}\) be a Boolean function. Then
  the following holds: 
  \begin{equation}
    \label{eq:2}
    \prob_{x,y \in \{-1,1\}^n}[\textsf{BLR}(f) \text{ accepts}] = \frac{1}{2} + \frac{1}{2}\sum_{S\subseteq \nset}\hat{f}(S)^3.
  \end{equation}
\end{lemma}

\begin{proof}
  The proof consists primarly in writing the probability term in a
  suitable way and using the Fourier expansion of \(f\). \\
  We write the indicator function \(\mathbb{1}_{\{\textsf{BLR}(f) \text{ accepts}\}}\) in the
  following way for \(x,y \in \{-1,1\}^n\)
  \[\mathbb{1}_{\{\textsf{BLR}(f) \text{ accepts}\}}(x,y) =
  \frac{1}{2} + \frac{1}{2}f(x)f(y)f(x\circ y).\]
  Hence we can express the right-hand side of (\ref{eq:2}) by
  \[\prob_{x,y \in \{-1,1\}^n}[\textsf{BLR}(f) \text{ accepts}] =
  \frac{1}{2} + \frac{1}{2}\expe_{x,y}[f(x)f(y)f(x\circ y)].\]
  Using the Fourier expansion of \(f\) we find by linearity: 
  \begin{align}
    \label{eq:3}
    \prob&_{x,y \in \{-1,1\}^n}[\textsf{BLR}(f) \text{ accepts}] \notag \\ 
    &=  \frac{1}{2} + \frac{1}{2} \sum_{S,T,U \subseteq \nset}\left(\hat{f}(S) \hat{f}(T) \hat{f}(U) \expe_{x,y}[\chi_S(x)\chi_T(y)\chi_U(x\circ y)]\right) \\
    &= \frac{1}{2} + \frac{1}{2} \sum_{S,T,U \subseteq \nset}\left(\hat{f}(S) \hat{f}(T) \hat{f}(U) \expe_{x,y}[\chi_S(x) \chi_T(y) \chi_U(x) \chi_U(y)]\right) \\
    &= \frac{1}{2} + \frac{1}{2} \sum_{S,T,U \subseteq \nset}\left(\hat{f}(S) \hat{f}(T) \hat{f}(U) \expe_{x,y}[\chi_{S\triangle U}(x) \cdot \chi_{T \triangle U}(y)]\right),
  \end{align}
  where we used the linearity of parity functions and the relation
  \(\chi_S(x) \cdot \chi_T(x) = \chi_{S\triangle T}(x)\).\\
  Since in the product measure \(\prob_{x,y}\) functions of \(x,y\)
  are independent we finally get 
  \begin{align}
    \label{eq:4}
    \prob&_{x,y \in \{-1,1\}^n}[\textsf{BLR}(f) \text{ accepts}] \notag \\ 
    &= \frac{1}{2} + \frac{1}{2} \sum_{S,T,U \subseteq \nset}\left(\hat{f}(S) \hat{f}(T) \hat{f}(U) \cdot \expe_x[\chi_{S\triangle U}(x)] \expe_y[\chi_{T \triangle U}(y)]\right) \\
    &= \frac{1}{2} + \frac{1}{2} \sum_{S,T,U \subseteq \nset}\left(\hat{f}(S) \hat{f}(T) \hat{f}(U) \cdot \langle \chi_S, \chi_U \rangle \cdot \langle \chi_T, \chi_U \rangle \right) \\
    &= \frac{1}{2} + \frac{1}{2} \sum_{S \subseteq \nset} \hat{f}(S)^3, 
  \end{align}
  since the parity functions are a orthonormal basis of all Boolean
  functions which means in particular
  \[\langle \chi_S, \chi_U \rangle \cdot \langle \chi_T, \chi_U \rangle = 
  \begin{cases}
    1 & \text{if } S = U = T, \\
    0 & \text{else}.
  \end{cases}
\]
\end{proof}


\begin{theorem}[Local testability of linearity]
  The property of being linear is locally testable. 
\end{theorem}

\emph{Here the proof is still missing! State that
  \(\prob[\textsf{BLR}(f) \text{ accepts}] < 1 - \eps\)}

\begin{proof}
  If \(f = \chi_S\) for some \(S\subseteq \nset\) then by Definition
  \ref{def:1} and the subsequent discussion \(f(x\circ y) = f(x) \cdot
  f(y)\) for all \(x,y \in \{-1,1\}^n\). \\
  Let now \(f\) be \(\eps\)-far from being linear and assume for the
  sake of contradiction 
  \[\prob_{x,y}\left[\textsf{BLR}(f) \text{ accepts}\right] \geq
  (1-\eps). \]
  Then using Lemma \ref{lem:1} we have \(1-\eps \leq \frac{1}{2} +
  \frac{1}{2} \sum_{S\subseteq \nset}\hat{f}(S)^3\) and consequently
  \begin{align}
    1 - 2\eps &\leq \sum_{S\subseteq \nset}\hat{f}(S)^3 \\
    &\leq \left(\max_{S\subseteq \nset} \hat{f}(S)\right)\cdot \sum_{S\subseteq \nset}\hat{f}(S)^2 \\
    &= \left(\max_{S\subseteq \nset} \hat{f}(S)\right) \cdot 1,  \label{al:1}
  \end{align}
  where we used Parseval's theorem in (\ref{al:1}). \\
  Hence there exists a subset \(T \subseteq \nset\) such that 
  \[1 - 2\eps \leq \expe_{x\in\{_1,1\}^n}\left[f(x)\chi_T(x)\right].\]
  But this implies 
  \[1 - \eps \leq \prob_{x\in\{_1,1\}^n}[f(x) = \chi_T(x)], \]
  which is a contradiction of \(f\) being \(\eps\)-far from being
  linear. 
\end{proof}

\begin{theorem}[Testability of linearity]
  The property of being linear is testable with
  \(\mathcal{O}\left(\frac{1}{\eps}\right)\) queries. 
\end{theorem}

\emph{Here the proof is still missing!}

\section{Dictator testing of Boolean functions}
\label{sec:dict-test-bool}


\end{document}
