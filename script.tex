\documentclass[a4paper]{article}

%--------------------------------------- Packages
\usepackage{mathtools}
\usepackage{amssymb}
\usepackage{amsthm}
\usepackage{enumitem}
\usepackage{bbold}
\usepackage{draftwatermark}

%--------------------------------------- Definitions
\newcommand{\prob}{\mathbf{P}}
\newcommand{\expe}{\mathbf{E}}
\newcommand{\real}{\mathbb{R}}
\newcommand{\prty}{\mathcal{P}}
\newcommand{\nset}{[n]}
\newcommand{\eps}{\varepsilon}
\newcommand{\boldx}{\boldsymbol{x}}
\newcommand{\boldy}{\boldsymbol{y}}
\newcommand{\bolde}{\boldsymbol{e}}

%--------------------------------------- Environments
\theoremstyle{plain}
\newtheorem{theorem}{Theorem}
\newtheorem{lemma}{Lemma}
\theoremstyle{definition}
\newtheorem{definition}{Definition}
\theoremstyle{remark}
\newtheorem*{remark*}{Remark}

%--------------------------------------- Some settings
\SetWatermarkScale{4}

%--------------------------------------- Document
\begin{document}
\title{Testing of Boolean functions}
\author{Pascal Huber}
\date{\today}
\maketitle

\section{Motivation}
\label{sec:motivation}

\section{Reminder}
\label{sec:reminder}

First we like to recall some basic results that we have already
covered in the previous course sessions and that we will need in our
investigation of linearity testing. \\

\noindent We consider functions from the hypercube \(\Omega_n \coloneqq
\{-1,1\}^n\) into the real numbers (we denote elements of \(\{-1,1\}^n
\) by boldfaced variables, e.g. \(\boldx\)). A function \(f: \Omega_n
\rightarrow \real\) is called \emph{Boolean function} if it takes values only in
\(\{-1,1\}\). 

\paragraph{Parity functions}
\label{sec:parity-functions}

% We recall two important types of Boolean functions:\\
We recall an important type of Boolean functions. Let \(\nset
\coloneqq \left\{1,2, \dots, n\right\}\) and let \(S\subseteq
\nset\). The \emph{parity function over S} is defined by 
\[ \chi_S(\boldx) =  \boldx_S \coloneqq \prod_{i \in S} x_i, \]
where we use the convention: \(\chi_\emptyset(\boldx) = \prod_{i  \in
  \emptyset} x_i = 1\) for all \(\boldx \in \{-1,1\}^n\). 

% \item The parity functions over the sets \(S = \{i\}\) are called
%   \emph{dictator functions}. They depend solely on the \(i\)-th bit,
%   i.e. 
%   \[\chi_i \coloneqq \chi_{\{i\}} = x_i.\] 
% \end{enumerate}

\paragraph{Great notational switch}
\label{sec:great-notat-switch}

Since we will treat linearity testing of Boolean functions it is in
some cases more suitable to replace the setting above by an
additive version (this is sometimes called the \emph{``great
  notational switch''}).\\
For this we replace the bit-set \(\{-1,1\}\) by \(\{0,1\}\)
(substituting \(1 \leadsto 0 \) and \(-1 \leadsto 1\)).  
Moreover the multiplication in \(\{-1,1\}\) is replaced by the
addition in \(\mathbb{F}_2\). 
In this setting a Boolean function has the form 
\[f: \{0,1\}^n \rightarrow \{0,1\}.\]
Note that the parity functions are in this setting just the functions of
the form \(\chi_S(\boldx) = \sum_{i\in S} x_i\) for some \(S\subseteq
\nset\). \\

\noindent Unless stated otherwise we will use the multiplicative setting, (i.e.
\(\Omega_n = \{-1,1\}^n\)). 


\paragraph{Correlation of Boolean functions}
\label{sec:corr-bool-funct}

\noindent The measurable space \(\left(\Omega_n, \mathcal{P}(\Omega_n)\right)\)
is equipped with the uniform probability measure \[\prob_{x \in
  \{-1,1\}^n} \coloneqq \left(\frac{1}{2} \delta_{-1} +
  \frac{1}{2}\delta_1\right)^{\otimes n}.\]
Furthermore we define the corresponding \(L^2\)-inner product (called \emph{correlation}) for two
functions \(f,g: \Omega_n \rightarrow \real\) on this space by 
\[\langle f,g\rangle \coloneqq \expe_{\boldx \in
  \{-1,1\}^n}\left[f(\boldx)g(\boldx)\right] = \frac{1}{2^n}\sum_{\boldx \in \{-1,1\}^n} f(\boldx)g(\boldx).\]
Note that if \(f\) and \(g\) are Boolean functions then one always
gets \(\langle f, g\rangle \in [-1,1]\) and \(\|f\| \coloneqq
\sqrt{\langle f, f\rangle } = 1\).  

\paragraph{Results from Fourier analysis of Boolean functions}
\label{sec:results-from-fourier}

Before introducing the notions of testability of Boolean functions we
recall the most important results from Fourier analysis of Boolean
functions. 
\begin{itemize}
\item The parity functions \(\left(\chi_S\right)_{S \subseteq \nset}\)
  form an orthonormal basis (w.r.t. the correlation) on the space of all functions from  \(\Omega_n\) to the real numbers, in particular 
  \[\langle \chi_S, \chi_T\rangle =  \begin{cases}
    1 & S=T \\
    0 & \text{else}
  \end{cases}\]
\item Consequently every Boolean function \(f\) has a unique
  representation of the form 
  \begin{equation}
    \label{eq:1}
    f = \sum_{S\subseteq \nset} \hat{f}(S) \chi_S.
  \end{equation}
  The coefficients \(\hat{f}(S)\) are called \emph{Fourier
    coefficients} and (\ref{eq:1}) is called the \emph{Fourier
    expansion} of \(f\).
\item \emph{Plancharel's theorem:} Let \(f,g: \Omega_n \rightarrow
  \real\). Then \(\langle f, g\rangle = \sum_{S\subseteq \nset}
  \hat{f}(S)\hat{g}(S)\).
\item \emph{Parseval's theorem:} Let \(f: \Omega_n \rightarrow
  \real\). Then \(\langle f, f\rangle = \sum_{S\subseteq \nset}
  \hat{f}(S)^2\). Especially, if \(f\) is Boolean one has the identity \(\sum_{S\subseteq \nset}
  \hat{f}(S)^2 = 1\). 
\end{itemize}


\section{Linearity of Boolean functions and basic notions of property testing}
\label{sec:line-bool-funct}

\subsection{Linearity of Boolean functions}
\label{sec:line-bool-funct-1}

In order to analyse linearity testing of Boolean functions (see
section \ref{sec:line-test-bool}) it is a
good start to define what it means for a Boolean
function to be linear. \\ 

\begin{remark*}
For simplicity we use in this subsection exclusively the additive
notation introduced in section \ref{sec:reminder} (\(\{-1,1\}\)
replaced by \(\{0,1\}\)). Of course the
analogous results hold also for the multiplicative setting.  
\end{remark*}

\begin{definition}[Linearity of Boolean functions] \label{def:1}
  A Boolean function \(f: \{0,1\}^n \rightarrow \{0,1\}\) is called \emph{linear}
  iff one of the following statements hold: 
  \begin{enumerate}[label= (\roman*)]
  \item For all \(\boldx,\boldy \in \{0,1\}^n\) one has
    \(f(\boldx+\boldy) = f(\boldx) + f(\boldy)\). 
  \item \(f\) is a parity function, i.e. there exists \(S \subseteq
    \nset\) such that for all \(\boldx \in \{0,1\}^n\) one has \(f(\boldx) =
    \sum_{i \in S}x_i\). 
  \end{enumerate}
\end{definition}

\begin{remark*}
  \noindent In multiplicative notation Definition \ref{def:1} reads as
  follows:
  \begin{enumerate}[label= (\roman*)]
  \item For all \(\boldx,\boldy \in \{-1,1\}^n\) one has
    \(f(\boldx\circ \boldy) = f(\boldx)\cdot f(\boldy)\), where
    \(\boldx\circ \boldy \coloneqq (x_1y_1, \dots, x_ny_n)\).
  \item There exists \(S \subseteq \nset\) such that for all
    \(\boldx\in \{-1,1\}^n\) one has \(f(\boldx) = \prod_{i\in S}
    x_i\).
  \end{enumerate}
\end{remark*}

\noindent In order to make sure that Definition
\ref{def:1} is well defined we have to show that both statements of
the definition are equivalent: \\
\underline{(ii) \(\Rightarrow\) (i):} is straightforward since 
\[\chi_S(\boldx+\boldy) = \sum_{i\in S} (x_i + y_i) = \sum_{i\in S} x_i +
\sum_{i\in S} y_i = \chi_S(\boldx) + \chi_S(\boldy).\]
\underline{(i) \(\Rightarrow\) (ii):} uses the representation \(\boldx =
x_1 \bolde_1 + \dots + x_n \bolde_n\) where \(\bolde_i \coloneqq
(0,\dots,0,1,0,\dots,0)\) with the non-zero bit at the \(i\)-th
position. Then by iterating (i) one gets 
\[f(x) = f\left(\sum_{i=1}^{n} x_i \bolde_i\right) = \sum_{i=1}^{n} x_i f(\bolde_i) =
\sum_{i\in S} x_i, \]
where \(S \coloneqq \left\{i\in \nset : f(\bolde_i) = 1 \right\}\). \\

\noindent Thus we see that the linear Boolean functions are exactly
the parity functions. 

\subsection{Approximate linearity and basic notions of property testing}
\label{sec:appr-line-basic}

In the following we introduce the basic notions of property testing
using the example of linearity testing. 

\paragraph{Approximate linearity}
\label{sec:appr-line}

The requirement to be linear on \(\{0,1\}^n\) is rather expensive to
check (one has to do all \(2^n\) queries) and hence for many
applications too restrictive. That's why one considers a more relaxed
concept of linearity: \\

\noindent A Boolean function \(f:\{0,1\}^n \rightarrow \{0,1\}\) is called
\emph{approximate linear} iff
\begin{enumerate}[label= (\roman*')]
\item for ``most'' pairs \(\boldx,\boldy \in \{0,1\}^n\) one has \(f(\boldx+\boldy) = f(\boldx)
  + f(\boldy)\).
\item there exist \(S \subseteq \nset\) s.t. for ``most'' \(\boldx\in
  \{0,1\}^n\) one has \(f(\boldx) = \chi_S(\boldx)\). 
\end{enumerate}

\noindent Analogously the multiplicative statements can be defined. \\ 

\noindent (We will soon define more precisely what is meant by ``most''.)\\

\noindent As before we would like both definitions to be equivalent
and one easily sees by copying the above argument that (ii') indeed
implies (i'). The converse implication on the other hand is much less
obvious. \\

\noindent A main result of this course will be that in fact (i') implies
(ii'). In order to show this we will use the results of the Fourier
analysis extensively. But before dwelling on the proof we want to put
the problem in the more general context of property testing. Therefore
we will need some definitions. 

\begin{remark*}
  From now on we will only use the multiplicative notation. 
\end{remark*}

\begin{definition}[Property of Boolean functions] \label{def:2}
  A \emph{property of Boolean functions} is a subset \(\prty\) of the
  set of all Boolean functions. We say that a Boolean function \(f\) \emph{has property \(\prty\)} if \(f
  \in \prty\). 
\end{definition}

\begin{definition}[\(\eps\)-far and \(\eps\)-close] \label{def:3}
  \begin{enumerate}[label=(\roman*)]
  \item Two Boolean functions \(f,g\) are \(\eps\)-close if they agree
    on a \((1-\eps)\)-fraction of \(\{-1,1\}^n\), i.e. 
    \[\prob_{\boldx\in\{-1,1\}^n}\left[f(\boldx) = g(\boldx)\right] \geq (1-\eps).\]
    Otherwise they are \emph{\(\eps\)-far}. 
  \item A Boolean function \(f\) is \emph{\(\eps\)-close} to having property
    \(\prty\) if there exists some function \(g\in \prty\) such that
    \(f\) and \(g\) are \(\eps\)-close. 
  \end{enumerate}
\end{definition}

\begin{remark*}
  Note that by definition of the correlation, two Boolean functions
  are \(\eps\)-close if and only if 
  \[\expe_{\boldx \in \{-1,1\}^n}\left[f(\boldx)g(\boldx)\right] \geq
  (1- 2\eps).\]  
\end{remark*}

\noindent In our case we are interested in the property of being
linear, i.e. \(\prty_{lin} \coloneqq \{\chi_S: S \subseteq
\nset\}\) and with Definition \ref{def:3} (ii') can be reformulated as
being \(\eps\)-close to \(\prty_{lin}\). \\

\noindent How can (i') be understood in this context? It is favorable to
interpret it as a \emph{property test}: 

\begin{definition}[\textsf{BLR} linearity test (Blum, Luby,
  Rubinfeld)] \label{def:4}
  Given blackbox access to a Boolean function \(f\) do the following
  steps: 
  \begin{enumerate}
  \item Pick \(\boldx\) and \(\boldy\) independently and uniformly at random
    from \(\{-1,1\}^n\).
  \item Query \(f\) on \(\boldx\), \(\boldy\) and \(\boldx+\boldy\).
  \item ``Accept'' iff \(f(\boldx)f(\boldy)f(\boldx \circ \boldy) = 1\). 
  \end{enumerate}
\end{definition}

\noindent Using this definition (i') can be understood in the sense
that the probability of \textsf{BLR} accepting the function \(f\) is
large, more precisely \[\prob_{x,y \in \{-1,1\}^n}\left[\textsf{BLR}(f) \text{
    \textit{accepts} }\right] \geq (1-\eps).\]
(Here \(\prob_{x,y \in \{-1,1\}^n}\) denotes the product measure on
\(\{-1,1\}^n \times \{-1,1\}^n\).)\\

\noindent If we can prove that (i') implies (ii') then we have
shown that for the linearity property there exists a querying
algorithm making only \(3\) queries such that whenever \(f\) is
\(\eps\)-far from being linear then the algorithm accepts \(f\) with
probability smaller than \(1-\eps\). \\
The existence of such a querying algorithm can also be shown for other
properties and motivates the following definition: 

\begin{definition}[Locally testable property] \label{def:5}
  A property \(\prty\) of Boolean functions is called \emph{locally
    testable} if there exists a randomized querying algorithm
  \(\mathcal{T}\) making at most \(\mathcal{O}(1)\) queries such that: 
  \begin{enumerate}[label=(\roman*)]
  \item If \(f \in \prty\) then \(\prob[\mathcal{T}(f) \text{ \textit{accepts}}] = 1\).
  \item If \(f\) is \(\eps\)-far from having property \(\prty\) then
    \(\prob[\mathcal{T}(f) \text{ \textit{accepts}}] \leq 1 - \Omega(\eps)\).
  \end{enumerate}
\end{definition}

\noindent A more general definition of testability which relates the
closeness to the property in question with the number of queries is
due to Rubinfeld and Sudan:  

\begin{definition}[Testable property] \label{def:6}
  A property \(\prty\) of Boolean functions is \emph{testable with
    \(q(\eps)\) queries} if there exists a randomized algorithm
  \(\mathcal{T}\) (which gets \(\eps\) as input) such that for all
  \(\eps > 0\) it makes \(q(\eps)\) random queries and satisfies:
  \begin{enumerate}[label=(\roman*)]
  \item If \(f \in \prty\) then \(\prob[\mathcal{T}(f) \text{
      \textit{accepts}}] \geq \frac{2}{3}\).
  \item If \(f\) is \(\eps\)-far from \(\prty\) then \(\prob[\mathcal{T}(f) \text{
      \textit{accepts}}] \leq \frac{1}{3}\).
  \end{enumerate}
\end{definition}

\begin{remark*}
  The choice of the bounds \(\frac{2}{3}\) and \(\frac{1}{3}\) is
  somewhat arbitrary and can be boosted to \(1-\delta\) and \(\delta\)
  respectively with a few more queries. 
\end{remark*}

\noindent The aim of today's course will be to show that the property
\(\prty_{lin}\) of being linear is 
\begin{itemize}
\item locally testable (this is the implication (i') \(\Rightarrow\)
  (ii'))
\item testable with \(\mathcal{O}\left(\frac{1}{\eps}\right)\)
  queries. 
\end{itemize}


\section{Linearity testing of Boolean functions}
\label{sec:line-test-bool}

\subsection{Testability of linearity}
\label{sec:test-line}

We will prove the testability of linearity in three steps: 
\begin{enumerate}
\item Express the ``acceptance probability'' of the \textsf{BLR}-test
  for an arbitrary Boolean function \(f\) in terms of its Fourier
  coefficients.
\item Prove local testability of linearity using this representation.
\item Prove testability by executing the \textsf{BLR}-test multiple
  times. 
\end{enumerate}

\begin{lemma} \label{lem:1}
  Let \(f: \{-1,1\}^n \rightarrow \{-1,1\}\) be a Boolean function. Then
  the following holds: 
  \begin{equation}
    \label{eq:2}
    \prob_{\boldx,\boldy \in \{-1,1\}^n}[\textsf{BLR}(f) \text{ \textit{accepts}}] = \frac{1}{2} + \frac{1}{2}\sum_{S\subseteq \nset}\hat{f}(S)^3.
  \end{equation}
\end{lemma}

\begin{proof}
  The proof consists primarly in writing the probability term in a
  suitable way and using the Fourier expansion of \(f\). \\
  We write the indicator function \(\mathbb{1}_{\{\textsf{BLR}(f) \text{ \textit{accepts}}\}}\) in the
  following way for \(\boldx,\boldy \in \{-1,1\}^n\)
  \[\mathbb{1}_{\{\textsf{BLR}(f) \text{ \textit{accepts}}\}}(\boldx,\boldy) =
  \frac{1}{2} + \frac{1}{2}f(\boldx)f(\boldy)f(\boldx\circ \boldy).\]
  Hence we can express the left-hand side of (\ref{eq:2}) by
  \[\prob_{\boldx,\boldy \in \{-1,1\}^n}[\textsf{BLR}(f) \text{ \textit{accepts}}] =
  \frac{1}{2} + \frac{1}{2} \expe_{\boldx,\boldy}[f(\boldx) f(\boldy)
  f(\boldx\circ\boldy)].\] 
  Using the Fourier expansion of \(f\) we find by linearity of the expectation: 
  \begin{align}
    \label{eq:3}
    \prob&_{\boldx,\boldy \in \{-1,1\}^n}[\textsf{BLR}(f) \text{ \textit{accepts}}] \notag \\  
    &=  \frac{1}{2} + \frac{1}{2} \sum_{S,T,U \subseteq \nset}\left(\hat{f}(S) \hat{f}(T) \hat{f}(U) \expe_{\boldx,\boldy}[\chi_S(\boldx) \chi_T(\boldy) \chi_U(\boldx\circ \boldy)]\right) \\
    &= \frac{1}{2} + \frac{1}{2} \sum_{S,T,U \subseteq \nset}\left(\hat{f}(S) \hat{f}(T) \hat{f}(U) \expe_{\boldx,\boldy}[\chi_S(\boldx) \chi_T(\boldy) \chi_U(\boldx) \chi_U(\boldy)]\right),  
  \end{align}
  where we used the linearity of parity functions. \\
  Since with respect to the product measure \(\prob_{\boldx,\boldy}\), two functions
  \(f(\boldx),f(\boldy)\) are independent we finally get 
  \begin{align}
    \label{eq:4}
    \prob&_{\boldx,\boldy \in \{-1,1\}^n}[\textsf{BLR}(f) \text{ \textit{accepts}}] \notag \\ 
    &= \frac{1}{2} + \frac{1}{2} \sum_{S,T,U \subseteq \nset}\left(\hat{f}(S) \hat{f}(T) \hat{f}(U) \cdot \expe_{\boldx}[\chi_S(\boldx)\chi_U(\boldx)] \expe_{\boldy}[\chi_T(\boldy)\chi_U(\boldy)]\right) \\
    &= \frac{1}{2} + \frac{1}{2} \sum_{S,T,U \subseteq \nset}\left(\hat{f}(S) \hat{f}(T) \hat{f}(U) \cdot \langle \chi_S, \chi_U \rangle \cdot \langle \chi_T, \chi_U \rangle \right) \\
    &= \frac{1}{2} + \frac{1}{2} \sum_{S \subseteq \nset} \hat{f}(S)^3, 
  \end{align}
  since the parity functions are a orthonormal basis of all Boolean
  functions which means in particular
  \[\langle \chi_S, \chi_U \rangle \cdot \langle \chi_T, \chi_U \rangle = 
  \begin{cases}
    1 & \text{if } S = U = T, \\
    0 & \text{else}.
  \end{cases}
\]
\end{proof}


\begin{theorem}[Local testability of linearity] \label{thm:1}
  The property of being linear is locally testable using the
  \textsf{BLR}-test and 
  \[\prob[\textsf{BLR}(f) \text{ \textit{accepts}}] < 1 - \eps,\]
  for every \(f\) which is \(\eps\)-far from being linear. 
\end{theorem}

\begin{proof}
  If \(f = \chi_S\) for some \(S\subseteq \nset\) then the
  \textsf{BLR}-test accepts with probability one by Definition
  \ref{def:1} and the subsequent discussion.\\ 
  Let now \(f\) be \(\eps\)-far from being linear and assume for the
  sake of contradiction 
  \[\prob_{\boldx,\boldy}\left[\textsf{BLR}(f) \text{ \textit{accepts}}\right] \geq
  (1-\eps). \]
  Then using Lemma \ref{lem:1} we have \(1-\eps \leq \frac{1}{2} +
  \frac{1}{2} \sum_{S\subseteq \nset}\hat{f}(S)^3\) and consequently
  \begin{align}
    1 - 2\eps &\leq \sum_{S\subseteq \nset}\hat{f}(S)^3 \leq \left(\max_{S\subseteq \nset} \hat{f}(S)\right)\cdot \sum_{S\subseteq \nset}\hat{f}(S)^2 = \left(\max_{S\subseteq \nset} \hat{f}(S)\right),  \label{al:1}
  \end{align}
  where we used Parseval's theorem for the last equality. \\
  Hence there exists a subset \(T \subseteq \nset\) such that 
  \[1 - 2\eps \leq \expe_{\boldx\in\{-1,1\}^n}\left[f(x)\chi_T(x)\right].\]
  By the remark after Definition \ref{def:3} this implies that \(f\)
  is \(\eps\)-close to \(\chi_T\)which is a contradiction to our
  assumption that \(f\) is \(\eps\)-far from being linear. 
\end{proof}

\begin{theorem}[Testability of linearity] \label{thm:2}
  The property of being linear is testable with
  \(\mathcal{O}\left(\frac{1}{\eps}\right)\) queries. 
\end{theorem}

\begin{proof}
  We define the randomized query algorithm \(\mathcal{T}(f,\eps)\)
  taking \(f\) and \(\eps\) as arguments as follows: 
  \begin{itemize}
  \item Run \(\textsf{BLR}(f)\) \(2/\eps\) times,
    independently.
  \item Accept iff every \(\textsf{BLR}(f)\) accepts. 
  \end{itemize}
  Note that if \(f\) is linear then \(\mathcal{T}(f)\) accepts with
  probability one, since \(\textsf{BLR}(f)\) accepts in every case.\\
  Assume now that \(f\) is \(\eps\)-far from being linear. Since the
  \textsf{BLR} queries are independent from each other, we get from
  Theorem \ref{thm:1}
  \begin{align}
    \prob\left[\mathcal{T}(f) \text{ \textit{accepts}}\right] 
    &< \left(1-\eps\right)^{2/\eps} \\
    &= \left(\left(1-\frac{1}{(1/\eps)}\right)^{1/\eps}\right)^2.
  \end{align}
  Since the last term converges for \(\eps \rightarrow 0\) to
  \(e^{-2}\) we get for \(\eps\) small enough (e.g. \(\eps \leq
  \frac{1}{2}\)) 
  \[\prob\left[\mathcal{T}(f) \text{ \textit{accepts}}\right] < \frac{1}{3}.\]
\end{proof}


\subsection{Local decodability of linearity}
\label{sec:local-decod-line}

Theorem \ref{thm:2} gives us a possibility to determine with high
probability if a given Boolean function is \(\eps\)-close to a parity
function. \\
Still the question remains how we can guess the right parity function
using a minimum number of queries? The following result tells us that
there is an algorithm making only two queries and predicts with high
probability the right parity function; this property is called
\emph{decodability}. 

\begin{theorem}[Local docadiblity of linearity] \label{thm:3}
  The property \(\mathcal{P}_{lin}\) of being a parity function is
  \emph{locally docodable} with 2 queries, i.e. there exists a
  randomized 2-query algorithm \(\mathcal{T}\) having access to a
  Boolean function \(f:\{-1,1\}^n \rightarrow \{-1,1\}\) and taking
  strings \(\boldx \in \{-1,1\}^n\) as input such that: \\
  If \(f\) is \(\eps\)-close to a  parity function \(\chi_S\), then
  for any (!) \(\boldx \in \{-1,1\}^n\) one has    
  \begin{equation}
    \label{eq:5}
    \prob\left[\mathcal{T}(\boldx) = \boldx_S\right] \geq 1 - 2\eps. 
  \end{equation}
\end{theorem}

\begin{remark*}
Note that the probability in (\ref{eq:5}) refers to the internal
randomness of the algorithm \(\mathcal{T}(\boldx)\) for an arbitrary but
fixed \(\boldx \in \{-1,1\}^n\).\\
In particular one can repeat the test multiple times for the same
\(\boldx\) to boost the probability in (\ref{eq:5}). \\ 
This is very different from the statement that for \emph{almost
  every} \(\boldx \in \{-1,1\}^n\) the algorithm computes the correct
value (this would be denoted by  \(\prob_{\boldx \in
  \{-1,1\}^n}[\mathcal{T}(\boldx) = \chi_S(\boldx)] \geq 1 -2\eps \)
and is much more trivial (just query \(f\) on \(\boldx\)).\\
\end{remark*}

\begin{proof}
  Let \(\boldx\in \{-1,1\}^n\) be fixed. We define \(\mathcal{T}\) as
  follows: For the given \(\boldx\)
  \begin{itemize}
  \item Pick \(\boldy\in\{-1,1\}^n\) uniformly at random.
  \item Return \(\mathcal{T}(\boldx) \coloneqq f(\boldy)f(\boldx\circ \boldy)\). 
  \end{itemize}
  We make the following observation: Since \(\boldy\) is uniformly
  distributed and \(f\) is \(\eps\)-close to \(\chi_S\) one has
  \begin{equation}
    \label{eq:6}
    \prob_{\boldy\in\{-1,1\}^n}[f(\boldy) = \chi_S(\boldy)] \geq 1 - \eps. 
  \end{equation}
  Similary \(\boldx\circ \boldy\) is again uniformly distributed (but not
  independent of \(\boldy\)) and 
  \begin{equation}
    \label{eq:7}
    \prob_{\boldy\in\{-1,1\}^n}[f(\boldx\circ \boldy) = \chi_S(\boldx\circ \boldy)] \geq 1 - \eps. 
  \end{equation}
  Using the union bound and (\ref{eq:6}) and (\ref{eq:7}) we find 
  \begin{align}
    \label{eq:8}
    \prob_{\boldy}[\{f(\boldx\circ \boldy) &= \chi_S(\boldx\circ \boldy)\} \cap \{f(\boldy) = \chi_S(\boldy)\}] \\
    &= \prob_{\boldy}[f(\boldx\circ \boldy) = \chi_S(\boldx\circ \boldy)] + \prob_{\boldy}[f(\boldy) = \chi_S(\boldy)] \\ 
    &- \prob_{\boldy}[\{f(\boldx\circ \boldy) = \chi_S(\boldx\circ \boldy)\} \cup \{f(\boldy) = \chi_S(\boldy)\}] \\
    &\geq (1-\eps) + (1-\eps) - 1 \\
    &= 1-2\eps. 
  \end{align}
  But in this case that booth equalities hold we have 
  \[\mathcal{T}(\boldx) = f(\boldy)f(\boldx\circ \boldy) = \boldy_S\boldx_S\boldy_S = \boldx_S,\] 
  hence 
  \[\prob_{\boldy}[\mathcal{T}(\boldx) = \boldx_S] \geq 1 - 2\eps.\]
  \end{proof}


\end{document}
